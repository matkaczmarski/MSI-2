% ================= IF YOU HAVE QUESTIONS =======================
% Technical questions to bbob@lri.fr
% ===============================================================

%%%%%%%%%%%%%%%%%%%%%%%%%%%%    PREAMBLE   %%%%%%%%%%%%%%%%%%%%%%%%%%%%%%%%%%%%

\documentclass{article}

\usepackage{polski}
\usepackage[utf8]{inputenc}
\usepackage{color}

\begin{document}

Dane wejściowe:\\
PSO\_DE(\textit{JNIfgeneric} fgeneric, $int$ dim, $doubl$e maxfunevals, $Random$ rand), gdzie:\\
\textit{JNIfgeneric} fgeneric – klasa z danymi definiującymi problem,\\
$int$ dim – wymiar problemu,\\
$double$ maxfunevals – maksymalna liczba ewaluacji,\\
$Random$ rand – generator liczb losowych\\

%\newpage
Schemat działania algorytmu hybrydowego łączącego DE oraz PSO:\\
\textit{
\indent{przypisz losowo początkowe wartości: $x_i$, $v_i$ oraz $p_i$  i $p_g$ dla $i = 1,2,..., N$}\\
\textcolor{blue}{while}(!stop)\\
\{\\
\indent	\textcolor{blue}{for} i = 1 to N\\
\indent	\{\\
\indent\indent użyj DE do wyznaczenia nowego kandydata - u\\
\indent\indent \{\\
\indent\indent		wybierz losowo $r_1$, $r_2$, $r_3$, takie że  $i \neq r_1 \neq r_2 \neq r_3$\\
\indent\indent\indent 		$m_i = x_{r_1} + F \cdot  (x_{r_2} - x_{r_3})$\\
\indent\indent\indent		\textcolor{blue}{for} j = 1 to D\\
\indent\indent\indent		\{\\
\indent\indent\indent\indent			wybierz losowo $j_{rand}$ z przedziału $[1, D]$\\
\indent\indent\indent\indent			\textcolor{blue}{if} (rand() $<$ CR or j $==$ $j_{rand}$)\\
\indent\indent\indent\indent\indent				u[j] = $m_i$[j]\\
\indent\indent\indent\indent			\textcolor{blue}{else}\\
\indent\indent\indent\indent\indent				u[j] = $x_i$[j]\\
\indent\indent\indent		\}\\
\indent\indent \}\\
\indent\indent		\textcolor{blue}{if} (f(u) $<$ f($x_i$))\\
\indent\indent\indent			$x_i$ = u\\
\indent\indent		\textcolor{blue}{else}\\
\indent\indent		\{\\
\indent\indent\indent			użyj PSO do wyznaczenia nowego kandydata – $TX_i$\\
\indent\indent\indent			\{\\
\indent\indent\indent\indent				oblicz wektor prędkości cząsteczki $x_i$:\\
\indent\indent\indent\indent				$v_i = \omega \cdot v_i + c_1 \cdot r_1 \cdot (p_i - x_i) + c_2 \cdot r_2 \cdot (p_g - x_i)$\\
\indent\indent\indent\indent				$TX_i = x_i + v_i$\\
\indent\indent\indent			\}\\
\indent\indent\indent			if (f($TX_i$) $<$ f($x_i$))\\
\indent\indent\indent\indent				$x_i$ = $TX_i$\\
\indent\indent	\}\\
\indent\indent		\textcolor{blue}{if} (f($x_i$) $<$ f($p_i$))\\
\indent\indent\indent			$p_i$ = $x_i$\\
\indent\indent		\textcolor{blue}{if} (f($x_i$) $<$ f($p_g$))\\
\indent\indent\indent			$p_g$ = $x_i$\\
\indent	\}\\
\}
}

\newpage
Schemat działania modyfikacji algorytmu hybrydowego łączącego DE oraz PSO:\\\\
\textit{
\indent{przypisz losowo początkowe wartości: $x_i$, $v_i$ oraz $p_i$  i $p_g$ dla $i = 1,2,..., N$}\\
\textcolor{blue}{while}(!stop)\\
\{\\
\indent	\textcolor{blue}{for} i = 1 to N\\
\indent	\{\\
\indent\indent użyj DE do wyznaczenia nowego kandydata - u\\
\indent\indent \{\\
\indent\indent		wybierz losowo $r_1$, $r_2$, $r_3$, takie że  $i \neq r_1 \neq r_2 \neq r_3$\\
\indent\indent\indent 		$m_i = x_{r_1} + F \cdot  (x_{r_2} - x_{r_3})$\\
\indent\indent\indent		\textcolor{blue}{for} j = 1 to D\\
\indent\indent\indent		\{\\
\indent\indent\indent\indent			wybierz losowo $j_{rand}$ z przedziału $[1, D]$\\
\indent\indent\indent\indent			\textcolor{blue}{if} (rand() $<$ CR or j $==$ $j_{rand}$)\\
\indent\indent\indent\indent\indent				u[j] = $m_i$[j]\\
\indent\indent\indent\indent			\textcolor{blue}{else}\\
\indent\indent\indent\indent\indent				u[j] = $x_i$[j]\\
\indent\indent\indent		\}\\
\indent\indent \}\\
\indent\indent		\textcolor{blue}{if} (f(u) $<$ f($x_i$))\\
\indent\indent \{\\
\indent\indent\indent			$v_i$ = u - $x_i$\\
\indent\indent\indent			$x_i$ = u\\
\indent\indent \}\\
\indent\indent		\textcolor{blue}{else}\\
\indent\indent		\{\\
\indent\indent\indent			użyj PSO do wyznaczenia nowego kandydata – $TX_i$\\
\indent\indent\indent			\{\\
\indent\indent\indent\indent				oblicz wektor prędkości cząsteczki $x_i$:\\
\indent\indent\indent\indent				$v_i = \omega \cdot v_i + c_1 \cdot r_1 \cdot (p_i - x_i) + c_2 \cdot r_2 \cdot (p_g - x_i)$\\
\indent\indent\indent\indent				$TX_i = x_i + v_i$\\
\indent\indent\indent			\}\\
\indent\indent\indent			if (f($TX_i$) $<$ f($x_i$))\\
\indent\indent\indent\indent				$x_i$ = $TX_i$\\
\indent\indent	\}\\
\indent\indent		\textcolor{blue}{if} (f($x_i$) $<$ f($p_i$))\\
\indent\indent\indent			$p_i$ = $x_i$\\
\indent\indent		\textcolor{blue}{if} (f($x_i$) $<$ f($p_g$))\\
\indent\indent\indent			$p_g$ = $x_i$\\
\indent	\}\\
\}
}

% The following two commands are all you need in the
% initial runs of your .tex file to
% produce the bibliography for the citations in your paper.
\bibliographystyle{abbrv}
\bibliography{bbob} % bbob.bib is the name of the bib file in this case
% You must have a proper ".bib" file
%  and remember to run:
% latex bibtex latex latex
% to resolve all references
\end{document}

