\providecommand{\bbobECDFslegend}[1]{
Bootstrapped empirical cumulative distribution of the number of objective function evaluations divided by dimension (FEvals/DIM) for 50 targets in $10^{[-8..2]}$ for all functions and subgroups in #1-D. The ``best 2009'' line corresponds to the best \ERT\ observed during BBOB 2009 for each single target.
}
\providecommand{\bbobppfigslegend}[1]{
Expected running time (\ERT\ in number of $f$-evaluations 
                as $\log_{10}$ value), divided by dimension for target function value $10^{-8}$ 
                versus dimension. Slanted grid lines indicate quadratic scaling with the dimension. Different symbols correspond to different algorithms given in the legend of #1. Light symbols give the maximum number of function evaluations from the longest trial divided by dimension. Black stars indicate a statistically better result compared to all other algorithms with $p<0.01$ and Bonferroni correction number of dimensions (six).  
Legend: 
{\color{NavyBlue}$\circ$}:\algorithmA
, {\color{red}$\triangledown$}:\algorithmB
, {\color{Goldenrod}$\star$}:\algorithmC
, {\color{VioletRed}$\Box$}:\algorithmD
}
\providecommand{\bbobpptablesmanylegend}[1]{%
    Expected running time (ERT in number of function 
    evaluations) divided by the respective best ERT measured during BBOB-2009 in
    #1.
    The ERT and in braces, as dispersion measure, the half difference between 90 and 
    10\%-tile of bootstrapped run lengths appear for each algorithm and 
    %
    target, the corresponding best ERT
    in the first row. The different target \Df-values are shown in the top row. 
    \#succ is the number of trials that reached the (final) target $\fopt + 10^{-8}$.
    %
    The median number of conducted function evaluations is additionally given in 
    \textit{italics}, if the target in the last column was never reached. 
    Entries, succeeded by a star, are statistically significantly better (according to
    the rank-sum test) when compared to all other algorithms of the table, with
    $p = 0.05$ or $p = 10^{-k}$ when the number $k$ following the star is larger
    than 1, with Bonferroni correction by the number of instances.
    }
\providecommand{\algorithmA}{PSO DE}
\providecommand{\algorithmB}{PSODEm}
\providecommand{\algorithmC}{PSODEr}
\providecommand{\algorithmD}{DE}
