\providecommand{\bbobECDFslegend}[1]{
Bootstrapped empirical cumulative distribution of the number of objective function evaluations divided by dimension (FEvals/DIM) for all functions and subgroups in #1-D. The targets are chosen from $10^{[-8..2]}$ such that the bestGECCO2009 artificial algorithm just not reached them within a given budget of $k$ $\times$ DIM, with $k\in \{0.5, 1.2, 3, 10, 50\}$. The ``best 2009'' line corresponds to the best \ERT\ observed during BBOB 2009 for each selected target.
}
\providecommand{\bbobppfigslegend}[1]{
Expected running time (\ERT\ in number of $f$-evaluations 
                as $\log_{10}$ value) divided by dimension versus dimension. The target function value 
                is chosen such that the bestGECCO2009 artificial algorithm just failed to achieve 
                an \ERT\ of $10\times\DIM$. Different symbols correspond to different algorithms given in the legend of #1. Light symbols give the maximum number of function evaluations from the longest trial divided by dimension. Black stars indicate a statistically better result compared to all other algorithms with $p<0.01$ and Bonferroni correction number of dimensions (six).  
Legend: 
{\color{NavyBlue}$\circ$}:\algorithmA
, {\color{red}$\triangledown$}:\algorithmB
}
\providecommand{\bbobppscatterlegend}[1]{
Expected running time (\ERT\ in $\log_{10}$ of number of function evaluations) 
    of \algorithmA\ ($y$-axis) versus \algorithmB\ ($x$-axis) for $8$ runlength-based target 
    function values for budgets between $0.5\times\DIM$ and $50\times\DIM$ evaluations. 
    Each runlength-based target $f$-value is chosen such that the \ERT{}s of the 
    bestGECCO2009 artificial algorithm for the given and a slightly easier 
    target bracket the reference budget. Markers on the upper or right edge indicate that the respective target
    value was never reached. Markers represent dimension: 
    2:{\color{cyan}+}, 
    3:{\color{green!45!black}$\triangledown$}, 
    5:{\color{blue}$\star$}, 
    10:$\circ$,
    20:{\color{red}$\Box$}, 
    40:{\color{magenta}$\Diamond$}. 
}
\providecommand{\bbobpptablestwolegend}[1]{
%
    Expected running time (ERT in number of function 
    evaluations) divided by the respective best ERT measured during BBOB-2009 in
    dimensions 5 (left) and 20 (right).
    The ERT and in braces, as dispersion measure, the half difference between 90 and 
    10\%-tile of bootstrapped run lengths appear for each algorithm and 
    %
    run-length based target, the corresponding best ERT
    (preceded by the target \Df-value in \textit{italics}) in the first row. 
    \#succ is the number of trials that reached the target value of the last column.
    %
    The median number of conducted function evaluations is additionally given in 
    \textit{italics}, if the target in the last column was never reached. 
    1:\algorithmAshort\ is \algorithmA\ and 2:\algorithmBshort\ is \algorithmB.
    Bold entries are statistically significantly better compared to the other algorithm,
    with $p=0.05$ or $p=10^{-k}$ where $k\in\{2,3,4,\dots\}$ is the number
    following the $\star$ symbol, with Bonferroni correction of #1.
    A $\downarrow$ indicates the same tested against the best algorithm of BBOB-2009.
    
}
\providecommand{\bbobpprldistrlegendtwo}[1]{
%
    Empirical cumulative distributions (ECDF)
    of run lengths and speed-up ratios in 5-D (left) and 20-D (right).
    Left sub-columns: ECDF of
    the number of function evaluations divided by dimension $D$
    (FEvals/D) %
    to fall below $\fopt+\Df$ for
    \algorithmA\ ($\circ$) and \algorithmB\ ($\triangledown$) where \Df\ is the
    target just not reached by the GECCO-BBOB-2009 best algorithm within a budget of
    $k\times\DIM$ evaluations, with $k$ being the value in the legend. 
    Right sub-columns:
    ECDF of FEval ratios of \algorithmA\ divided by \algorithmB\ for run-length-based
    targets; all
    trial pairs for each function. Pairs where both trials failed are disregarded,
    pairs where one trial failed are visible in the limits being $>0$ or $<1$. The
    legends indicate the target budget of $k\times\DIM$ evaluations and,
    after the colon, the number of functions that were solved in at least one trial
    (\algorithmA\ first).
}
\providecommand{\algorithmA}{PSO DE modified}
\providecommand{\algorithmB}{PSO DE}
\providecommand{\bbobECDFslegend}[1]{
Bootstrapped empirical cumulative distribution of the number of objective function evaluations divided by dimension (FEvals/DIM) for all functions and subgroups in #1-D. The targets are chosen from $10^{[-8..2]}$ such that the bestGECCO2009 artificial algorithm just not reached them within a given budget of $k$ $\times$ DIM, with $k\in \{0.5, 1.2, 3, 10, 50\}$. The ``best 2009'' line corresponds to the best \ERT\ observed during BBOB 2009 for each selected target.
}
\providecommand{\bbobppfigslegend}[1]{
Expected running time (\ERT\ in number of $f$-evaluations 
                as $\log_{10}$ value) divided by dimension versus dimension. The target function value 
                is chosen such that the bestGECCO2009 artificial algorithm just failed to achieve 
                an \ERT\ of $10\times\DIM$. Different symbols correspond to different algorithms given in the legend of #1. Light symbols give the maximum number of function evaluations from the longest trial divided by dimension. Black stars indicate a statistically better result compared to all other algorithms with $p<0.01$ and Bonferroni correction number of dimensions (six).  
Legend: 
{\color{NavyBlue}$\circ$}:\algorithmA
, {\color{red}$\triangledown$}:\algorithmB
}
\providecommand{\bbobppscatterlegend}[1]{
Expected running time (\ERT\ in $\log_{10}$ of number of function evaluations) 
    of \algorithmA\ ($y$-axis) versus \algorithmB\ ($x$-axis) for $8$ runlength-based target 
    function values for budgets between $0.5\times\DIM$ and $50\times\DIM$ evaluations. 
    Each runlength-based target $f$-value is chosen such that the \ERT{}s of the 
    bestGECCO2009 artificial algorithm for the given and a slightly easier 
    target bracket the reference budget. Markers on the upper or right edge indicate that the respective target
    value was never reached. Markers represent dimension: 
    2:{\color{cyan}+}, 
    3:{\color{green!45!black}$\triangledown$}, 
    5:{\color{blue}$\star$}, 
    10:$\circ$,
    20:{\color{red}$\Box$}, 
    40:{\color{magenta}$\Diamond$}. 
}
\providecommand{\bbobpptablestwolegend}[1]{
%
    Expected running time (ERT in number of function 
    evaluations) divided by the respective best ERT measured during BBOB-2009 in
    dimensions 5 (left) and 20 (right).
    The ERT and in braces, as dispersion measure, the half difference between 90 and 
    10\%-tile of bootstrapped run lengths appear for each algorithm and 
    %
    run-length based target, the corresponding best ERT
    (preceded by the target \Df-value in \textit{italics}) in the first row. 
    \#succ is the number of trials that reached the target value of the last column.
    %
    The median number of conducted function evaluations is additionally given in 
    \textit{italics}, if the target in the last column was never reached. 
    1:\algorithmAshort\ is \algorithmA\ and 2:\algorithmBshort\ is \algorithmB.
    Bold entries are statistically significantly better compared to the other algorithm,
    with $p=0.05$ or $p=10^{-k}$ where $k\in\{2,3,4,\dots\}$ is the number
    following the $\star$ symbol, with Bonferroni correction of #1.
    A $\downarrow$ indicates the same tested against the best algorithm of BBOB-2009.
    
}
\providecommand{\bbobpprldistrlegendtwo}[1]{
%
    Empirical cumulative distributions (ECDF)
    of run lengths and speed-up ratios in 5-D (left) and 20-D (right).
    Left sub-columns: ECDF of
    the number of function evaluations divided by dimension $D$
    (FEvals/D) %
    to fall below $\fopt+\Df$ for
    \algorithmA\ ($\circ$) and \algorithmB\ ($\triangledown$) where \Df\ is the
    target just not reached by the GECCO-BBOB-2009 best algorithm within a budget of
    $k\times\DIM$ evaluations, with $k$ being the value in the legend. 
    Right sub-columns:
    ECDF of FEval ratios of \algorithmA\ divided by \algorithmB\ for run-length-based
    targets; all
    trial pairs for each function. Pairs where both trials failed are disregarded,
    pairs where one trial failed are visible in the limits being $>0$ or $<1$. The
    legends indicate the target budget of $k\times\DIM$ evaluations and,
    after the colon, the number of functions that were solved in at least one trial
    (\algorithmA\ first).
}
\providecommand{\algorithmA}{PSO DE modified}
\providecommand{\algorithmB}{PSO DE}
\providecommand{\algorithmAshort}{PSO}
\providecommand{\algorithmBshort}{PSO}
\providecommand{\algorithmAshort}{PSO}
\providecommand{\algorithmBshort}{PSO}
